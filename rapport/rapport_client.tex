\documentclass[11pt]{article}
%Gummi|065|=)
\usepackage[francais]{babel}
\usepackage[T1]{fontenc}
\usepackage{fancyvrb}
\usepackage{xcolor}

\definecolor{Zgris}{rgb}{0.87,0.85,0.85}

\newsavebox{\BBbox}
\newenvironment{DDbox}[1]{
\begin{lrbox}{\BBbox}\begin{minipage}{\linewidth}}
{\end{minipage}\end{lrbox}\noindent\colorbox{Zgris}{\usebox{\BBbox}} \\
[.5cm]}

\title{\textbf{Explication du code : Partie client}}
\author{Florian Cormée\\
		Hugo Duarte}
\date{}

\begin{document}

\maketitle

\section{Interaction Graphique et Communication Serveur}

On s'interresse aux interactions de l'utilisateur avec l'environnement graphique et la communication de ces infos au serveur.\\

\begin{DDbox}{\linewidth}
\begin{Verbatim}
case SDL_MOUSEBUTTONDOWN:
	SDL_GetMouseState(&mx, &my);
\end{Verbatim}
\end{DDbox}
\footnote{sh13.c l.270}
Dans un premier temps en récupère la position de la sourie sur l'interface lors d'un clic, puis on regarde la position de la sourie plus certain etat du système. 
\vspace{1em}

\begin{DDbox}{\linewidth}
\begin{Verbatim}
if ((mx < 200) && (my < 50) && (connectEnabled == 1)) {

	sprintf(sendBuffer, "C %s %d %s", 
	gClientIpAddress, gClientPort, gName);

	sendMessageToServer(gServerIpAddress,
	gServerPort, sendBuffer);
	connectEnabled = 0;
} ...
\end{Verbatim}
\end{DDbox}
\footnote{sh13.c l.270}

Ici par exemple, on vérifie si le bouton connecté est activé et si l'utilisateur a cliqué dessus. On envoi par la suite un message au serveur avec le token \emph{C} pour informer de la connection du joueur, enfin on désactive le bouton connecté.

\begin{DDbox}{\linewidth}
\begin{Verbatim}
else if ((mx >= 500) && (mx < 700) && (my >= 350) &&
(my < 450) && (goEnabled == 1)) {
	printf("go! joueur=%d objet=%d guilt=%d\n",
	joueurSel, objetSel, guiltSel);
	sprintf(sendBuffer, "G %d %d", gId, guiltSel);

	sendMessageToServer(gServerIpAddress, 
	gServerPort, sendBuffer);
} ...
\end{Verbatim}
\end{DDbox}
\footnote{sh13.c l.270}


\begin{DDbox}{\linewidth}
\begin{Verbatim}
else if ((objetSel != -1) && (joueurSel == -1)) {
	sprintf(sendBuffer, "O %d %d", gId, objetSel);

	sendMessageToServer(gServerIpAddress,
	gServerPort, sendBuffer);
} ...
\end{Verbatim}
\end{DDbox}
\footnote{sh13.c l.270}


\begin{DDbox}{\linewidth}
\begin{Verbatim}
} else if ((objetSel != -1) && (joueurSel != -1)) {
	sprintf(sendBuffer, "S %d %d %d", gId,
	joueurSel, objetSel);

	sendMessageToServer(gServerIpAddress,
	gServerPort, sendBuffer);
} ...
\end{Verbatim}
\end{DDbox}
\footnote{sh13.c l.270}

\section{Traitement des consignes du serveur}

\begin{DDbox}{\linewidth}
\begin{Verbatim}
switch (gbuffer[0]) {
	case 'I':
		sscanf(gbuffer + 2, "%d", &gId);
		printf("Mon Id c'est %d\n", gId);
		break;
	case 'L':
		sscanf(gbuffer + 2, "%s %s %s %s", *gNames,
		*(gNames + 1), *(gNames + 2), *(gNames + 3));
		break;
	case 'D':
		sscanf(gbuffer + 2, "%d %d %d", b, b + 1,
		b + 2);
		break;
	case 'M': {
		size_t num;
		sscanf(gbuffer + 2, "%ld", &num);
		printf("Num c'est %ld\n", num);
		goEnabled = (num == gId);
		printf("GoEnabled : %d\n", goEnabled);
	} break;
	case 'V': {
		int joueur, objet, valeur;
		sscanf(gbuffer + 2, "%d %d %d", &joueur, &objet,
		&valeur);
		if (tableCartes[joueur][objet] == -1 ||
		tableCartes[joueur][objet] == 100) {
			tableCartes[joueur][objet] = valeur;}
	} break;
	case 'W': {
		int idcc;
		sscanf(gbuffer + 2, "%d %d", &gWinnerId, &idcc);
		goEnabled = 0;
		printf(
		"Elementaire my dear %s, c'etait bien %s !\n",
		gNames[gWinnerId], nbnoms[idcc]);
	} break;
\end{Verbatim}
\end{DDbox}
\footnote{sh13.c l.346}

\section{Option : Création du chat}

\begin{DDbox}{\linewidth}
\begin{Verbatim}
void *fn_chat(void *arg) {
	char msg[128];
	char data[256];

	while (1) {
		fgets(msg, 128, stdin);
		if (gId >= 0) {
			sprintf(data, "T %d %s", gId, msg);
			sendMessageToServer(gServerIpAddress,
			gServerPort, data); 
		}
	}
}
\end{Verbatim}
\end{DDbox}
\footnote{sh13.c l.129}

Le but de cette fonction est d'offrir la possibilité aux joueurs de communiquer entre eux. Elle récupère le message saisit par l'utilisateur dans \emph{msg} et l'envoie au serveur dans \emph{data} avec la forme suivante : "\emph{T gId msg}"avec le token \emph{T} pour le serveur, l'indice de l'utilisateur \emph{gId} et le message \emph{msg}.\\

Par la suite, le serveur transmet le message à tous les joueurs

\begin{DDbox}{\linewidth}
\begin{Verbatim}
case 'T': {
	int playerId = -1;
	char msg[128];
	char com;
	memset(msg, '\0', 128);

	sscanf(gbuffer, "%c %d %s", &com, &playerId, msg);

	if (playerId == -1) {
		puts("[T] Identifiant invalide");
	} else if (strlen(msg) == 0) {
		printf(
		"[%c] %s essaie d'envoyer un message vide.",
		com, gNames[playerId]);
	} else {
		printf("%s > %s\n", gNames[playerId], msg);
	}
} break;
\end{Verbatim}
\end{DDbox}
\footnote{sh13.c l.406}

Dans le cas où le serveur communique une message de la part d'un joueur, on recupère le message dans \emph{gbuffer}, on vérifie que l'identifiant de l'expediteur est valide et que le message n'est pas vide. Enfin, on l'affiche aux différents destinataires dans le terminal.

\end{document}
