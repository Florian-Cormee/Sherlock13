\subsection{Fonctionnement du client}

\subsubsection{Présentation de l'algorithme}

Au lancement, le client initialise ses variables, se connecte au serveur et
génère l'interface graphique. On rentre ensuite dans la boucle principale qui
est découpé en trois partie. Dans un premier temps, le programme analyse les
interactions que l'utilisateur a avec la fenêtre graphique et envoie les
commandes associées au serveur. Ensuite, si une requête est reçu du serveur, il
exécute les commandes associées. Enfin, il met à jour la fenêtre graphique à
l'aide des nouvelles informations reçu. La boucle est quittée quand la fenêtre
est fermée et les différents objets graphiques sont détruits.

\subsubsection{Répartition du code}

Afin de rendre le code source du client plus lisible, nous l'avons découpé en
de multiples fichiers et fonctions. Ce découpage est thématique. Nous avons
aussi essayé de ne pas avoir des fonctions de plus de 100 lignes ainsi que des
fichiers de moins de 500 lignes de code.

Ainsi, les fichiers \verb=cartes.h= et \verb=cartes.c= regroupent les fonctions
et les variables globales liées aux cartes.

La communication a été abstraite avec les fichiers \verb=com.h= et \verb=com.c=.
Ces fichiers proposent des fonctions d'envoi de message au serveur.

Les fichiers \verb=gui.h= et \verb=gui.c= gèrent les fonctions liées à
l'interface graphique ainsi que toutes les variables liées à l'environnement
graphique. De cette façon, \verb=sh13.c= ne contient que la boucle principale.
