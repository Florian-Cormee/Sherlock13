\subsection{Fonctionnement du serveur}

\subsubsection{Présentation de l'algorithme}

Tout d'abord, le serveur initialise ses variables. Les adresses des clients sont
\verb|localhost| et le port est initialisé à une valeur impossible. Le serveur
mélange les cartes et rempli sa table de vérité. Ainsi, le coupable est tiré au
sors et le programme comptabilise la quantité de chaque symbole que possêdera
chacun des joueurs.

Suite à cela, le programme prépare un socket en tant que serveur lié au port
passé en argument. Dès lors, le programme attend la réception d'un message.

À la réception d'un message le programme acceptera différentes commandes selon
son état. Comme le montre la figure \ref{fig:uml_sequence}, à l'état initiale,
le serveur n'accepte que des demandes de connection. Une fois les quatres
connections établies, le serveur envoie ses cartes et sa ligne de la table de
vérité à chaque joueur. Puis il annonce le début du tour du premier joueur.
Enfin, le programme change d'état pour gérer les commandes liées au déroulement
du jeu.

Dans ce second état, le serveur accepte les commandes d'accusation et de
questionnement. Quand l'une d'elles est réceptionnée, le programme vérifie que
l'expéditeur est bien le joueur actif, dans le cas contraire, un message est
affiché dans la console et la requête est refusée. Une fois la requête
accomplie, le serveur annonce l'identifiant du joueur suivant. L'opération se
répète jusqu'à se qu'un jour parvienne à démasquer le coupable ou que tous les
joueurs est été éliminé, ceci se produit lorsqu'une fausse accusation est
lancée.

\subsubsection{Répartition du code}

Afin de rendre le code source du serveur plus lisible, nous l'avons découpé en
de multiples fichiers et fonctions. Se découpage est thématique. Nous avons
aussi essayé de ne pas avoir des fonctions de plus de 100 lignes ainsi que des
fichiers de moins de 500 lignes de code.

Ainsi, les fichiers \verb=cartes.h= et \verb=cartes.c= regroupes les fonctions
et les variables globales liées aux cartes.

La communication a été abstraite avec les fichiers \verb=com.h= et \verb=com.c=.
Ces fichiers proposent des fonctions d'envoie de message à un client ou à tous
les clients. Ils regroupent aussi les variables globales contenant les adresses
des clients.

Les fichiers \verb=msg.h= et \verb=msg.c= ajoute de l'abstraction vis-à-vis du
formatage d'un message. De cette façon, \verb=server.c= ne contient que la
boucle principale, la machine à état et les callback y correspondant.
