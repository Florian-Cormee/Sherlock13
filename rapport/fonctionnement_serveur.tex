\subsection{Fonctionnement du serveur}

Tout d'abord, le serveur initialise ses variables. Les adresses des clients sont
\verb|localhost| et le port est initialisé à une valeur impossible. Le serveur
mélange les cartes et rempli sa table de vérité. Ainsi, le coupable est tiré au
sors et le programme comptabilise la quantité de chaque symbole que possêdera
chacun des joueurs.

Suite à cela, le programme prépare un socket en tant que serveur lié au port
passé en argument. Dès lors, le programme attend la réception d'un message.

À la réception d'un message le programme acceptera différentes commandes selon
son état. Comme le montre la figure \ref{fig:uml_sequence}, à l'état initiale,
le serveur n'accepte que des demandes de connection. Une fois les quatres
connections établies, le serveur envoie ses cartes et sa ligne de la table de
vérité à chaque joueur. Puis il annonce le début du tour du premier joueur.
Enfin, le programme change d'état pour gérer les commandes liées au déroulement
du jeu.
